\documentclass[english]{article}

\newcommand{\class}[1]{{\texttt #1}}
\newcommand{\method}[1]{{\texttt #1}}
\newcommand{\package}[1]{{\textsf #1}}

\begin{document}

\title{Present status of the multi-type tree sampler}

\maketitle

\section{Introduction}

We discuss here the current status and future goals of the project
to implement a multi-type (or coloured) tree distribution sampler
within the BEAST framework. Such a sampler will enable us to infer
parameters under the assumption of sophisticated population genetics
and epidemiological models.

Currently, the sampler is implemented as an add-on called
\package{ColouredTree}. The eventual goal, however, is to integrate the
sampler into the production BEAST 2 code.


\section{Overview of framework}

To date, we have a working framework for the sampler in place.  The
best place to go for the nitty-gritty details of the sampler is the
JavaDoc output from the ColouredTree source.  What follows is
therefore only a cursory outline of the framework.


\subsection{The ColouredTree plugin}

The main class of the sampler is \class{ColouredTree}, which derives
from \class{Plugin}. This class is essentially just a wrapper around
a \class{Tree} input and four \class{Parameter} inputs which together
contain the information about the tree topology and the distribution
of migration/infection events along its branches.  Figure 1 contains a
schematic illustrating this structure.

Besides this primary function, \class{ColouredTree} contains public
methods necessary for obtaining information about the multi-type tree
it represents.  At the most basic level this includes methods which
return the number of migration events along each branch, as well as
the type and time of each of these events.

Noteably, \class{ColouredTree} does \emph{not} contain the methods
necessary for modifying any of the details of these events.  This is
due to the restriction imposed by BEAST that only methods belonging to
classes derived from \class{Operator} are permitted to alter instances
of \class{StateNode} such as \class{Parameter}.  For this reason,
many of these methods (which were originally contained in
\class{ColouredTree}) have been moved to \class{ColouredTreeOperator}.

There is one exception to this rule: the \method{initFromFlatTree()}
method, which avoids this difficulty by calling the more primitive
methods belonging to \class{ColouredTreeModifier} which is a special
subclass of \class{ColouredTreeOperator} constructed specifically for
this purpose.

\subsection{The ColouredTreeOperator plugin}

The abstract \class{ColouredTreeOperator} class is designed to be
subclassed by operators intended to modify \class{ColouredTree}
configurations.  It includes all of the methods one might otherwise
expect to see in \class{ColouredTree}, including primitive setters
like \method{setChangeColour()}, \method{setChangeTime()} and
\method{setChangeCount()}, togther with composite and more useful
methods such as \method{addChange()}.

Importantly, \class{ColouredTreeOperator} now also defines a method
\method{setRoot()} which should be used in place of the method of the
same name defined by the \class{Tree} class to change the root node of
a multi-type tree topology.  The reason behind this is that
\class{Tree.setRoot()} has recently begun to alter the ID numbers of
nodes in the tree to ensure that the root node is always given the ID
$n-1$.  The method \class{ColouredTreeOperator.setRoot()} updates the
data in \class{ColouredTree} to reflect this change.

As mentioned above, there is a special subclass of the abstract
\class{ColouredTreeOperator} named \class{ColouredTreeModifier} which
is intended to be directly instantiated and used in methods which need
to be able to modify \class{ColouredTree} objects but for which direct
inclusion in \class{ColouredTreeOperator} does not make sense from a
structural point of view.  This includes the
\class{ColouredTree.initFromFlatTree()} method mentioned previously,
as well as methods implementing \class{StateNodeInitialiser} such as
\class{StructuredCoalescentColouredTree}. (Ordinarily, a
\class{StateNodeInitialiser} sets the initial values of a
\class{StateNode} using its constructor.  This is not an ideal
approach to constructing complex multi-type tree objects, however,
which are much easier to construct in an incremental fashion.

\subsection{The ColouredTreeDistribution plugin}

The \class{ColouredTreeDistribution} plugin is another abstract class,
in this case designed to be subclassed by likelihoods and priors over
the multi-type tree space.  Only two such distributions presently
exist: a structured coalescent distribution named
\class{StructuredCoalescentLikelihood} and Denise's birth-death model
likelihood.

\subsection{The MigrationModel plugin}

This plugin is intended to be the equivalent of
\class{SubstitutionModel} for multi-type trees. Its primary contents
are a migration rate matrix and a vector of deme population sizes. It
also contains a set of helper methods which calculate eigenvalue
decompositions of the rate matrix and use this to calculate transition
probabilities between demes along lineages. It is derived from
\class{CalculationNode} as the eigenvalue decomposition must be
recalculated whenever the migration rates and/or population sizes are
altered.


\subsection{The ColouredTreeFromNewick plugin}

This plugin is useful for initialising \class{ColouredTree} objects
using a tree specification given in Newick format.  Colour changes
are specified using single-child nodes and the colours themselves are
specified as traits on the node below the branches they refer to.
Internally, this plugin uses the \class{TreeParser} plugin in
combination with \class{ColouredTree.initFromFlatTree()}.

\section{Operators for multi-type trees}

\subsection{Special considerations}

\subsection{Implemented operators}

\begin{enumerate}
\item ColouredWilsonBaldingRandom
\item ColouredWilsonBalding
\item ColouredTreeScale
\item ColouredSubtreeSlide
\item ColouredUniform
\end{enumerate}


\section{Work yet to do}
\end{document}
